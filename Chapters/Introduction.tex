% Introducción
\chapter*{Introduction} % Main chapter title

\label{Intro} % For referencing the chapter elsewhere, use \ref{Chapter1} 

\lhead{\emph{Introduction}} % This is for the header on each page - perhaps a shortened title

%----------------------------------------------------------------------------------------
%----------------------------------------------------------------------------------------
%	INTRODUCTION
%----------------------------------------------------------------------------------------

Rigidity phenomena called mathematicians attention because it uses the structure of the objects to describe morphisms between them. We want to give an answer to the rather vague question: \textit{How \textbf{common} is rigidity in the curve complex?}

The complex of curves appears naturally in the study of $Mod(S)$, the mapping class group of a surface $S$ which is a central object in contemporary mathematical research. The folkloric version of rigidity in this context will mean that if we consider $X$ and $Y$, under suitable conditions, then every homomorphism $Mod(X) \to Mod(Y)$ will be induced by a manipulation of the underlying surfaces.

To give fomal meaning to the word "common" we propose the use of probabilistic tools to model the curve complex. Complex topological spaces arise quite natural in a lot of scientific contexts. Probability theory provides different approaches to model those spaces; even in complex configurations, it can be possible, by doing approximations, to study topological invariants. Most of the times the main goal is to study the asymptotic behavior when a parameter of the model tends to infinity. In this sense stochastic topology can be thought as tool for topology in the same sense as statistical mechanics is used to study a macroscopic physical systems when the classical mechanics finds these systems very complicated to solve.

Stochastic topology finds its motivation in applied problems such as statistical mechanics, configuration spaces of mechanic links, robotics, animated simulations, manufacture, and so on. Nevertheless recent jobs have used its techniques to provide probabilistic analogs of very classical topology conjectures, like Whitehead conjecture \cite{Costa15}. This analogues can be interpreted like statistical evidence for the conjecture. The idea is to proof that the sentence is true (almost surely), under certain conditions in the probabilistic model.

In chapter one, we motivate the study of the curve complex $C(S)$, an abstract simplicial complex associated to a surface $S$. $C(S)$ encodes intersection patterns of simple closed curves in $S$ and it's highly related to $Mod(S)$. After reviewing the most important properties of these objects we'll be ready to introduce rigidity, a concept valid in graph theory with a meaningful interpretation in this context.

In the second chapter we'll study $C(S)$ in a probabilistic perspective. We'll explain how to choose the parameters that fit the proposed model, namely, the flag complex taken from an Erdös-Rényi graph.

A wide range of techniques had been used in this area, among them, algorithms which sample the search space or replicate certain deterministic phenomenon of interest. In the applied sense they have shown to be the most successful techniques, see \cite{Alcazar15} for example. In the theoretical results, like in \cite{Meshulam13}, they have allowed to determine whether the established conditions in the parameters of the spaces are sharp enough or whether the conjectures about certain properties are probably valid or not. In the last chapter we explain an stochastic algorithm built to do rigid expansions and we'll give the expected execution time depending on the parameters taken.



























